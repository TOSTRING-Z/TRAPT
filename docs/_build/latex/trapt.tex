%% Generated by Sphinx.
\def\sphinxdocclass{report}
\documentclass[letterpaper,10pt,english]{sphinxmanual}
\ifdefined\pdfpxdimen
   \let\sphinxpxdimen\pdfpxdimen\else\newdimen\sphinxpxdimen
\fi \sphinxpxdimen=.75bp\relax
\ifdefined\pdfimageresolution
    \pdfimageresolution= \numexpr \dimexpr1in\relax/\sphinxpxdimen\relax
\fi
%% let collapsible pdf bookmarks panel have high depth per default
\PassOptionsToPackage{bookmarksdepth=5}{hyperref}


\PassOptionsToPackage{warn}{textcomp}
\usepackage[utf8]{inputenc}
\ifdefined\DeclareUnicodeCharacter
% support both utf8 and utf8x syntaxes
  \ifdefined\DeclareUnicodeCharacterAsOptional
    \def\sphinxDUC#1{\DeclareUnicodeCharacter{"#1}}
  \else
    \let\sphinxDUC\DeclareUnicodeCharacter
  \fi
  \sphinxDUC{00A0}{\nobreakspace}
  \sphinxDUC{2500}{\sphinxunichar{2500}}
  \sphinxDUC{2502}{\sphinxunichar{2502}}
  \sphinxDUC{2514}{\sphinxunichar{2514}}
  \sphinxDUC{251C}{\sphinxunichar{251C}}
  \sphinxDUC{2572}{\textbackslash}
\fi
\usepackage{cmap}
\usepackage[T1]{fontenc}
\usepackage{amsmath,amssymb,amstext}
\usepackage{babel}



\usepackage{tgtermes}
\usepackage{tgheros}
\renewcommand{\ttdefault}{txtt}



\usepackage[Bjarne]{fncychap}
\usepackage{sphinx}

\fvset{fontsize=auto}
\usepackage{geometry}


% Include hyperref last.
\usepackage{hyperref}
% Fix anchor placement for figures with captions.
\usepackage{hypcap}% it must be loaded after hyperref.
% Set up styles of URL: it should be placed after hyperref.
\urlstyle{same}


\usepackage{sphinxmessages}




\title{TRAPT}
\date{Nov 15, 2024}
\release{0.1.7}
\author{Guorui Zhang}
\newcommand{\sphinxlogo}{\vbox{}}
\renewcommand{\releasename}{Release}
\makeindex
\begin{document}

\ifdefined\shorthandoff
  \ifnum\catcode`\=\string=\active\shorthandoff{=}\fi
  \ifnum\catcode`\"=\active\shorthandoff{"}\fi
\fi

\pagestyle{empty}
\sphinxmaketitle
\pagestyle{plain}
\sphinxtableofcontents
\pagestyle{normal}
\phantomsection\label{\detokenize{index::doc}}


\sphinxAtStartPar
November 15, 2024

\sphinxAtStartPar
\sphinxstylestrong{Title} TRAPT: A multi\sphinxhyphen{}stage fused deep learning framework for transcriptional
regulators prediction via integrating large\sphinxhyphen{}scale epigenomic data

\sphinxAtStartPar
\sphinxstylestrong{Maintainer} Guorui Zhang \textless{}\sphinxhref{mailto:mp798378522@gmail.com}{mp798378522@gmail.com}\textgreater{}

\sphinxAtStartPar
\sphinxstylestrong{Description} TRAPT is a multi\sphinxhyphen{}omics integration framework designed
for inferring transcriptional regulator activity from a set of query genes.
TRAPT employs a multi\sphinxhyphen{}stage fusion strategy to address the issues
of incomplete cis\sphinxhyphen{}regulatory profile coverage and TRBP problems.
By leveraging two\sphinxhyphen{}stage self\sphinxhyphen{}knowledge distillation to extract the activity
embedding of regulatory elements, TRAPT can predicts key regulatory
factors for sets of query genes through a fusion strategy.


\chapter{TRAPT.CalcSampleRPMatrix module}
\label{\detokenize{index:module-TRAPT.CalcSampleRPMatrix}}\label{\detokenize{index:trapt-calcsamplerpmatrix-module}}\index{module@\spxentry{module}!TRAPT.CalcSampleRPMatrix@\spxentry{TRAPT.CalcSampleRPMatrix}}\index{TRAPT.CalcSampleRPMatrix@\spxentry{TRAPT.CalcSampleRPMatrix}!module@\spxentry{module}}\index{dhs2gene() (in module TRAPT.CalcSampleRPMatrix)@\spxentry{dhs2gene()}\spxextra{in module TRAPT.CalcSampleRPMatrix}}

\begin{fulllineitems}
\phantomsection\label{\detokenize{index:TRAPT.CalcSampleRPMatrix.dhs2gene}}
\pysigstartsignatures
\pysiglinewithargsret{\sphinxcode{\sphinxupquote{TRAPT.CalcSampleRPMatrix.}}\sphinxbfcode{\sphinxupquote{dhs2gene}}}{\emph{\DUrole{n}{args}}, \emph{\DUrole{n}{sample}}}{}
\pysigstopsignatures
\sphinxAtStartPar
Calculate the Epi regulatory potential score.

\sphinxAtStartPar
Parameters:
args : argparse.Namespace
\begin{quote}

\sphinxAtStartPar
Global parameters.
\end{quote}
\begin{description}
\sphinxlineitem{sample}{[}str{]}
\sphinxAtStartPar
Epi sample name.

\sphinxlineitem{vec}{[}np.array{]}
\sphinxAtStartPar
Epi PRE score.

\end{description}
\begin{quote}\begin{description}
\sphinxlineitem{Returns}
\sphinxAtStartPar
Epi sample name, and Epi\sphinxhyphen{}RP score.

\end{description}\end{quote}

\end{fulllineitems}



\chapter{TRAPT.CalcTRAUC module}
\label{\detokenize{index:module-TRAPT.CalcTRAUC}}\label{\detokenize{index:trapt-calctrauc-module}}\index{module@\spxentry{module}!TRAPT.CalcTRAUC@\spxentry{TRAPT.CalcTRAUC}}\index{TRAPT.CalcTRAUC@\spxentry{TRAPT.CalcTRAUC}!module@\spxentry{module}}\index{CalcTRAUC (class in TRAPT.CalcTRAUC)@\spxentry{CalcTRAUC}\spxextra{class in TRAPT.CalcTRAUC}}

\begin{fulllineitems}
\phantomsection\label{\detokenize{index:TRAPT.CalcTRAUC.CalcTRAUC}}
\pysigstartsignatures
\pysiglinewithargsret{\sphinxbfcode{\sphinxupquote{class\DUrole{w}{  }}}\sphinxcode{\sphinxupquote{TRAPT.CalcTRAUC.}}\sphinxbfcode{\sphinxupquote{CalcTRAUC}}}{\emph{\DUrole{n}{args}}, \emph{\DUrole{n}{RP\_Matrix\_TR\_Sample}}, \emph{\DUrole{n}{w}}}{}
\pysigstopsignatures
\sphinxAtStartPar
Bases: \sphinxcode{\sphinxupquote{object}}

\sphinxAtStartPar
Calculate the area under the curve (AUC) for each TR curve.
\index{args (TRAPT.CalcTRAUC.CalcTRAUC attribute)@\spxentry{args}\spxextra{TRAPT.CalcTRAUC.CalcTRAUC attribute}}

\begin{fulllineitems}
\phantomsection\label{\detokenize{index:TRAPT.CalcTRAUC.CalcTRAUC.args}}
\pysigstartsignatures
\pysigline{\sphinxbfcode{\sphinxupquote{args}}}
\pysigstopsignatures
\sphinxAtStartPar
Global parameters for TRAPT.
\begin{quote}\begin{description}
\sphinxlineitem{Type}
\sphinxAtStartPar
{\hyperref[\detokenize{index:TRAPT.Tools.Args}]{\sphinxcrossref{TRAPT.Tools.Args}}}

\end{description}\end{quote}

\end{fulllineitems}

\index{RP\_Matrix\_TR\_Sample (TRAPT.CalcTRAUC.CalcTRAUC attribute)@\spxentry{RP\_Matrix\_TR\_Sample}\spxextra{TRAPT.CalcTRAUC.CalcTRAUC attribute}}

\begin{fulllineitems}
\phantomsection\label{\detokenize{index:TRAPT.CalcTRAUC.CalcTRAUC.RP_Matrix_TR_Sample}}
\pysigstartsignatures
\pysigline{\sphinxbfcode{\sphinxupquote{RP\_Matrix\_TR\_Sample}}}
\pysigstopsignatures
\sphinxAtStartPar
The sum of TR\sphinxhyphen{}RP scores and D\sphinxhyphen{}RP scores.
\begin{quote}\begin{description}
\sphinxlineitem{Type}
\sphinxAtStartPar
anndata.AnnData

\end{description}\end{quote}

\end{fulllineitems}

\index{w (TRAPT.CalcTRAUC.CalcTRAUC attribute)@\spxentry{w}\spxextra{TRAPT.CalcTRAUC.CalcTRAUC attribute}}

\begin{fulllineitems}
\phantomsection\label{\detokenize{index:TRAPT.CalcTRAUC.CalcTRAUC.w}}
\pysigstartsignatures
\pysigline{\sphinxbfcode{\sphinxupquote{w}}}
\pysigstopsignatures
\sphinxAtStartPar
U\sphinxhyphen{}RP scores.
\begin{quote}\begin{description}
\sphinxlineitem{Type}
\sphinxAtStartPar
np.array

\end{description}\end{quote}

\end{fulllineitems}

\subsubsection*{Notes}

\sphinxAtStartPar
The input is the RP matrix, and the calculation is performed as follows:
\begin{equation*}
\begin{split}IRP = (TRRP + DRP) \times URP\end{split}
\end{equation*}\index{get\_auc() (TRAPT.CalcTRAUC.CalcTRAUC static method)@\spxentry{get\_auc()}\spxextra{TRAPT.CalcTRAUC.CalcTRAUC static method}}

\begin{fulllineitems}
\phantomsection\label{\detokenize{index:TRAPT.CalcTRAUC.CalcTRAUC.get_auc}}
\pysigstartsignatures
\pysiglinewithargsret{\sphinxbfcode{\sphinxupquote{static\DUrole{w}{  }}}\sphinxbfcode{\sphinxupquote{get\_auc}}}{\emph{\DUrole{n}{params}}}{}
\pysigstopsignatures
\sphinxAtStartPar
Parallel computing module.

\sphinxAtStartPar
Parameters:
i : int
\begin{quote}

\sphinxAtStartPar
The i\sphinxhyphen{}th row of the I\sphinxhyphen{}RP matrix.
\end{quote}
\begin{description}
\sphinxlineitem{j}{[}int{]}
\sphinxAtStartPar
Default is 0.

\sphinxlineitem{labels}{[}np.array{]}
\sphinxAtStartPar
Gene vector.

\sphinxlineitem{vec}{[}np.array{]}
\sphinxAtStartPar
I\sphinxhyphen{}RP vector.

\end{description}
\begin{quote}\begin{description}
\sphinxlineitem{Returns}
\sphinxAtStartPar
i, j, and the AUC score of the i\sphinxhyphen{}th TR.

\end{description}\end{quote}

\end{fulllineitems}

\index{iter\_params() (TRAPT.CalcTRAUC.CalcTRAUC method)@\spxentry{iter\_params()}\spxextra{TRAPT.CalcTRAUC.CalcTRAUC method}}

\begin{fulllineitems}
\phantomsection\label{\detokenize{index:TRAPT.CalcTRAUC.CalcTRAUC.iter_params}}
\pysigstartsignatures
\pysiglinewithargsret{\sphinxbfcode{\sphinxupquote{iter\_params}}}{\emph{\DUrole{n}{gene\_vec}}, \emph{\DUrole{n}{trunk}}}{}
\pysigstopsignatures
\sphinxAtStartPar
Parallel parameter module.

\sphinxAtStartPar
Parameters:
gene\_vec : np.array
\begin{quote}

\sphinxAtStartPar
Gene vector.
\end{quote}
\begin{description}
\sphinxlineitem{trunk}{[}int{]}
\sphinxAtStartPar
Number of blocks.

\end{description}
\begin{quote}\begin{description}
\sphinxlineitem{Returns}
\sphinxAtStartPar
An iterator.

\end{description}\end{quote}

\end{fulllineitems}

\index{run() (TRAPT.CalcTRAUC.CalcTRAUC method)@\spxentry{run()}\spxextra{TRAPT.CalcTRAUC.CalcTRAUC method}}

\begin{fulllineitems}
\phantomsection\label{\detokenize{index:TRAPT.CalcTRAUC.CalcTRAUC.run}}
\pysigstartsignatures
\pysiglinewithargsret{\sphinxbfcode{\sphinxupquote{run}}}{}{}
\pysigstopsignatures
\sphinxAtStartPar
TR auc calculation module execution entry point.
\begin{quote}\begin{description}
\sphinxlineitem{Returns}
\sphinxAtStartPar
A pd.DataFrame of AUC scores for TRs.

\end{description}\end{quote}

\end{fulllineitems}


\end{fulllineitems}



\chapter{TRAPT.CalcTRRPMatrix module}
\label{\detokenize{index:module-TRAPT.CalcTRRPMatrix}}\label{\detokenize{index:trapt-calctrrpmatrix-module}}\index{module@\spxentry{module}!TRAPT.CalcTRRPMatrix@\spxentry{TRAPT.CalcTRRPMatrix}}\index{TRAPT.CalcTRRPMatrix@\spxentry{TRAPT.CalcTRRPMatrix}!module@\spxentry{module}}\index{dhs2gene() (in module TRAPT.CalcTRRPMatrix)@\spxentry{dhs2gene()}\spxextra{in module TRAPT.CalcTRRPMatrix}}

\begin{fulllineitems}
\phantomsection\label{\detokenize{index:TRAPT.CalcTRRPMatrix.dhs2gene}}
\pysigstartsignatures
\pysiglinewithargsret{\sphinxcode{\sphinxupquote{TRAPT.CalcTRRPMatrix.}}\sphinxbfcode{\sphinxupquote{dhs2gene}}}{\emph{\DUrole{n}{params}}}{}
\pysigstopsignatures
\sphinxAtStartPar
Calculate the TR regulatory potential score.

\sphinxAtStartPar
Parameters:
args : argparse.Namespace
\begin{quote}

\sphinxAtStartPar
Global parameters.
\end{quote}
\begin{description}
\sphinxlineitem{sample}{[}str{]}
\sphinxAtStartPar
TR sample name.

\sphinxlineitem{vec}{[}np.array{]}
\sphinxAtStartPar
TR PRE score.

\end{description}
\begin{quote}\begin{description}
\sphinxlineitem{Returns}
\sphinxAtStartPar
TR sample name, and TR\sphinxhyphen{}RP score.

\end{description}\end{quote}

\end{fulllineitems}

\index{str2bool() (in module TRAPT.CalcTRRPMatrix)@\spxentry{str2bool()}\spxextra{in module TRAPT.CalcTRRPMatrix}}

\begin{fulllineitems}
\phantomsection\label{\detokenize{index:TRAPT.CalcTRRPMatrix.str2bool}}
\pysigstartsignatures
\pysiglinewithargsret{\sphinxcode{\sphinxupquote{TRAPT.CalcTRRPMatrix.}}\sphinxbfcode{\sphinxupquote{str2bool}}}{\emph{\DUrole{n}{v}}}{}
\pysigstopsignatures
\end{fulllineitems}



\chapter{TRAPT.CalcTRSampleRPMatrix module}
\label{\detokenize{index:module-TRAPT.CalcTRSampleRPMatrix}}\label{\detokenize{index:trapt-calctrsamplerpmatrix-module}}\index{module@\spxentry{module}!TRAPT.CalcTRSampleRPMatrix@\spxentry{TRAPT.CalcTRSampleRPMatrix}}\index{TRAPT.CalcTRSampleRPMatrix@\spxentry{TRAPT.CalcTRSampleRPMatrix}!module@\spxentry{module}}\index{CalcTRSampleRPMatrix (class in TRAPT.CalcTRSampleRPMatrix)@\spxentry{CalcTRSampleRPMatrix}\spxextra{class in TRAPT.CalcTRSampleRPMatrix}}

\begin{fulllineitems}
\phantomsection\label{\detokenize{index:TRAPT.CalcTRSampleRPMatrix.CalcTRSampleRPMatrix}}
\pysigstartsignatures
\pysiglinewithargsret{\sphinxbfcode{\sphinxupquote{class\DUrole{w}{  }}}\sphinxcode{\sphinxupquote{TRAPT.CalcTRSampleRPMatrix.}}\sphinxbfcode{\sphinxupquote{CalcTRSampleRPMatrix}}}{\emph{\DUrole{n}{library}\DUrole{o}{=}\DUrole{default_value}{\textquotesingle{}library\textquotesingle{}}}, \emph{\DUrole{n}{output}\DUrole{o}{=}\DUrole{default_value}{\textquotesingle{}library\textquotesingle{}}}, \emph{\DUrole{n}{type}\DUrole{o}{=}\DUrole{default_value}{\textquotesingle{}H3K27ac\textquotesingle{}}}}{}
\pysigstopsignatures
\sphinxAtStartPar
Bases: \sphinxcode{\sphinxupquote{object}}
\index{run() (TRAPT.CalcTRSampleRPMatrix.CalcTRSampleRPMatrix method)@\spxentry{run()}\spxextra{TRAPT.CalcTRSampleRPMatrix.CalcTRSampleRPMatrix method}}

\begin{fulllineitems}
\phantomsection\label{\detokenize{index:TRAPT.CalcTRSampleRPMatrix.CalcTRSampleRPMatrix.run}}
\pysigstartsignatures
\pysiglinewithargsret{\sphinxbfcode{\sphinxupquote{run}}}{}{}
\pysigstopsignatures
\end{fulllineitems}


\end{fulllineitems}



\chapter{TRAPT.DLFS module}
\label{\detokenize{index:module-TRAPT.DLFS}}\label{\detokenize{index:trapt-dlfs-module}}\index{module@\spxentry{module}!TRAPT.DLFS@\spxentry{TRAPT.DLFS}}\index{TRAPT.DLFS@\spxentry{TRAPT.DLFS}!module@\spxentry{module}}\index{CustomSigmoid (class in TRAPT.DLFS)@\spxentry{CustomSigmoid}\spxextra{class in TRAPT.DLFS}}

\begin{fulllineitems}
\phantomsection\label{\detokenize{index:TRAPT.DLFS.CustomSigmoid}}
\pysigstartsignatures
\pysiglinewithargsret{\sphinxbfcode{\sphinxupquote{class\DUrole{w}{  }}}\sphinxcode{\sphinxupquote{TRAPT.DLFS.}}\sphinxbfcode{\sphinxupquote{CustomSigmoid}}}{\emph{\DUrole{o}{*}\DUrole{n}{args}}, \emph{\DUrole{o}{**}\DUrole{n}{kwargs}}}{}
\pysigstopsignatures
\sphinxAtStartPar
Bases: \sphinxcode{\sphinxupquote{Layer}}
\index{call() (TRAPT.DLFS.CustomSigmoid method)@\spxentry{call()}\spxextra{TRAPT.DLFS.CustomSigmoid method}}

\begin{fulllineitems}
\phantomsection\label{\detokenize{index:TRAPT.DLFS.CustomSigmoid.call}}
\pysigstartsignatures
\pysiglinewithargsret{\sphinxbfcode{\sphinxupquote{call}}}{\emph{\DUrole{n}{x}}}{}
\pysigstopsignatures
\sphinxAtStartPar
This is where the layer’s logic lives.

\sphinxAtStartPar
The \sphinxtitleref{call()} method may not create state (except in its first
invocation, wrapping the creation of variables or other resources in
\sphinxtitleref{tf.init\_scope()}).  It is recommended to create state, including
\sphinxtitleref{tf.Variable} instances and nested \sphinxtitleref{Layer} instances,
\begin{quote}

\sphinxAtStartPar
in \sphinxtitleref{\_\_init\_\_()}, or in the \sphinxtitleref{build()} method that is
\end{quote}

\sphinxAtStartPar
called automatically before \sphinxtitleref{call()} executes for the first time.
\begin{quote}\begin{description}
\sphinxlineitem{Parameters}\begin{itemize}
\item {} 
\sphinxAtStartPar
\sphinxstyleliteralstrong{\sphinxupquote{inputs}} \textendash{} 
\sphinxAtStartPar
Input tensor, or dict/list/tuple of input tensors.
The first positional \sphinxtitleref{inputs} argument is subject to special rules:
\sphinxhyphen{} \sphinxtitleref{inputs} must be explicitly passed. A layer cannot have zero
\begin{quote}

\sphinxAtStartPar
arguments, and \sphinxtitleref{inputs} cannot be provided via the default value
of a keyword argument.
\end{quote}
\begin{itemize}
\item {} 
\sphinxAtStartPar
NumPy array or Python scalar values in \sphinxtitleref{inputs} get cast as
tensors.

\item {} 
\sphinxAtStartPar
Keras mask metadata is only collected from \sphinxtitleref{inputs}.

\item {} 
\sphinxAtStartPar
Layers are built (\sphinxtitleref{build(input\_shape)} method)
using shape info from \sphinxtitleref{inputs} only.

\item {} 
\sphinxAtStartPar
\sphinxtitleref{input\_spec} compatibility is only checked against \sphinxtitleref{inputs}.

\item {} 
\sphinxAtStartPar
Mixed precision input casting is only applied to \sphinxtitleref{inputs}.
If a layer has tensor arguments in \sphinxtitleref{*args} or \sphinxtitleref{**kwargs}, their
casting behavior in mixed precision should be handled manually.

\item {} 
\sphinxAtStartPar
The SavedModel input specification is generated using \sphinxtitleref{inputs}
only.

\item {} 
\sphinxAtStartPar
Integration with various ecosystem packages like TFMOT, TFLite,
TF.js, etc is only supported for \sphinxtitleref{inputs} and not for tensors in
positional and keyword arguments.

\end{itemize}


\item {} 
\sphinxAtStartPar
\sphinxstyleliteralstrong{\sphinxupquote{*args}} \textendash{} Additional positional arguments. May contain tensors, although
this is not recommended, for the reasons above.

\item {} 
\sphinxAtStartPar
\sphinxstyleliteralstrong{\sphinxupquote{**kwargs}} \textendash{} 
\sphinxAtStartPar
Additional keyword arguments. May contain tensors, although
this is not recommended, for the reasons above.
The following optional keyword arguments are reserved:
\sphinxhyphen{} \sphinxtitleref{training}: Boolean scalar tensor of Python boolean indicating
\begin{quote}

\sphinxAtStartPar
whether the \sphinxtitleref{call} is meant for training or inference.
\end{quote}
\begin{itemize}
\item {} 
\sphinxAtStartPar
\sphinxtitleref{mask}: Boolean input mask. If the layer’s \sphinxtitleref{call()} method takes a
\sphinxtitleref{mask} argument, its default value will be set to the mask
generated for \sphinxtitleref{inputs} by the previous layer (if \sphinxtitleref{input} did come
from a layer that generated a corresponding mask, i.e. if it came
from a Keras layer with masking support).

\end{itemize}


\end{itemize}

\sphinxlineitem{Returns}
\sphinxAtStartPar
A tensor or list/tuple of tensors.

\end{description}\end{quote}

\end{fulllineitems}


\end{fulllineitems}

\index{FeatureSelection (class in TRAPT.DLFS)@\spxentry{FeatureSelection}\spxextra{class in TRAPT.DLFS}}

\begin{fulllineitems}
\phantomsection\label{\detokenize{index:TRAPT.DLFS.FeatureSelection}}
\pysigstartsignatures
\pysiglinewithargsret{\sphinxbfcode{\sphinxupquote{class\DUrole{w}{  }}}\sphinxcode{\sphinxupquote{TRAPT.DLFS.}}\sphinxbfcode{\sphinxupquote{FeatureSelection}}}{\emph{\DUrole{n}{args}}, \emph{\DUrole{n}{data\_ad}}, \emph{\DUrole{n}{type}}}{}
\pysigstopsignatures
\sphinxAtStartPar
Bases: \sphinxcode{\sphinxupquote{object}}
\index{TSFS() (TRAPT.DLFS.FeatureSelection method)@\spxentry{TSFS()}\spxextra{TRAPT.DLFS.FeatureSelection method}}

\begin{fulllineitems}
\phantomsection\label{\detokenize{index:TRAPT.DLFS.FeatureSelection.TSFS}}
\pysigstartsignatures
\pysiglinewithargsret{\sphinxbfcode{\sphinxupquote{TSFS}}}{\emph{\DUrole{n}{X}}, \emph{\DUrole{n}{T}}}{}
\pysigstopsignatures
\sphinxAtStartPar
Teacher\sphinxhyphen{}Student Feature Selection.

\sphinxAtStartPar
Parameters:
X : np.array
\begin{quote}

\sphinxAtStartPar
Epi\sphinxhyphen{}RP matrix.
\end{quote}
\begin{description}
\sphinxlineitem{T}{[}str{]}
\sphinxAtStartPar
Input genes vector.

\end{description}
\begin{quote}\begin{description}
\sphinxlineitem{Returns}
\sphinxAtStartPar
Index values sorted by Epi sample weights, and Epi sample weights.

\end{description}\end{quote}

\end{fulllineitems}

\index{get\_act() (TRAPT.DLFS.FeatureSelection method)@\spxentry{get\_act()}\spxextra{TRAPT.DLFS.FeatureSelection method}}

\begin{fulllineitems}
\phantomsection\label{\detokenize{index:TRAPT.DLFS.FeatureSelection.get_act}}
\pysigstartsignatures
\pysiglinewithargsret{\sphinxbfcode{\sphinxupquote{get\_act}}}{\emph{\DUrole{n}{t}\DUrole{o}{=}\DUrole{default_value}{1}}}{}
\pysigstopsignatures
\sphinxAtStartPar
U\sphinxhyphen{}RP teacher model activation function.

\sphinxAtStartPar
Parameters:
t : float
\begin{quote}

\sphinxAtStartPar
Temperature value.
\end{quote}

\end{fulllineitems}

\index{get\_corr() (TRAPT.DLFS.FeatureSelection method)@\spxentry{get\_corr()}\spxextra{TRAPT.DLFS.FeatureSelection method}}

\begin{fulllineitems}
\phantomsection\label{\detokenize{index:TRAPT.DLFS.FeatureSelection.get_corr}}
\pysigstartsignatures
\pysiglinewithargsret{\sphinxbfcode{\sphinxupquote{get\_corr}}}{\emph{\DUrole{n}{v1}}, \emph{\DUrole{n}{v2}}}{}
\pysigstopsignatures
\sphinxAtStartPar
Correlation calculation.

\end{fulllineitems}

\index{get\_loss() (TRAPT.DLFS.FeatureSelection method)@\spxentry{get\_loss()}\spxextra{TRAPT.DLFS.FeatureSelection method}}

\begin{fulllineitems}
\phantomsection\label{\detokenize{index:TRAPT.DLFS.FeatureSelection.get_loss}}
\pysigstartsignatures
\pysiglinewithargsret{\sphinxbfcode{\sphinxupquote{get\_loss}}}{}{}
\pysigstopsignatures
\sphinxAtStartPar
U\sphinxhyphen{}RP teacher model loss function.

\end{fulllineitems}

\index{run() (TRAPT.DLFS.FeatureSelection method)@\spxentry{run()}\spxextra{TRAPT.DLFS.FeatureSelection method}}

\begin{fulllineitems}
\phantomsection\label{\detokenize{index:TRAPT.DLFS.FeatureSelection.run}}
\pysigstartsignatures
\pysiglinewithargsret{\sphinxbfcode{\sphinxupquote{run}}}{}{}
\pysigstopsignatures
\sphinxAtStartPar
Method execution entry point.
\begin{quote}\begin{description}
\sphinxlineitem{Returns}
\sphinxAtStartPar
A pd.DataFrame of U\sphinxhyphen{}RP scores for query Genes, and selected sample information.

\end{description}\end{quote}

\end{fulllineitems}

\index{sort\_by\_group() (TRAPT.DLFS.FeatureSelection method)@\spxentry{sort\_by\_group()}\spxextra{TRAPT.DLFS.FeatureSelection method}}

\begin{fulllineitems}
\phantomsection\label{\detokenize{index:TRAPT.DLFS.FeatureSelection.sort_by_group}}
\pysigstartsignatures
\pysiglinewithargsret{\sphinxbfcode{\sphinxupquote{sort\_by\_group}}}{\emph{\DUrole{n}{vec}}}{}
\pysigstopsignatures
\sphinxAtStartPar
Grouping function.

\end{fulllineitems}

\index{train() (TRAPT.DLFS.FeatureSelection method)@\spxentry{train()}\spxextra{TRAPT.DLFS.FeatureSelection method}}

\begin{fulllineitems}
\phantomsection\label{\detokenize{index:TRAPT.DLFS.FeatureSelection.train}}
\pysigstartsignatures
\pysiglinewithargsret{\sphinxbfcode{\sphinxupquote{train}}}{\emph{\DUrole{n}{X}}, \emph{\DUrole{n}{y}}}{}
\pysigstopsignatures
\sphinxAtStartPar
U\sphinxhyphen{}RP model training entry function.

\sphinxAtStartPar
Parameters:
X : np.array
\begin{quote}

\sphinxAtStartPar
Epi\sphinxhyphen{}RP matrix.
\end{quote}
\begin{description}
\sphinxlineitem{y}{[}str{]}
\sphinxAtStartPar
Input genes vector.

\end{description}
\begin{quote}\begin{description}
\sphinxlineitem{Returns}
\sphinxAtStartPar
U\sphinxhyphen{}RP model.

\end{description}\end{quote}

\end{fulllineitems}


\end{fulllineitems}

\index{SparseGroupLasso (class in TRAPT.DLFS)@\spxentry{SparseGroupLasso}\spxextra{class in TRAPT.DLFS}}

\begin{fulllineitems}
\phantomsection\label{\detokenize{index:TRAPT.DLFS.SparseGroupLasso}}
\pysigstartsignatures
\pysiglinewithargsret{\sphinxbfcode{\sphinxupquote{class\DUrole{w}{  }}}\sphinxcode{\sphinxupquote{TRAPT.DLFS.}}\sphinxbfcode{\sphinxupquote{SparseGroupLasso}}}{\emph{\DUrole{n}{l1}\DUrole{o}{=}\DUrole{default_value}{0.01}}, \emph{\DUrole{n}{l2}\DUrole{o}{=}\DUrole{default_value}{0.01}}, \emph{\DUrole{n}{groups}\DUrole{o}{=}\DUrole{default_value}{None}}}{}
\pysigstopsignatures
\sphinxAtStartPar
Bases: \sphinxcode{\sphinxupquote{Regularizer}}
\index{get\_config() (TRAPT.DLFS.SparseGroupLasso method)@\spxentry{get\_config()}\spxextra{TRAPT.DLFS.SparseGroupLasso method}}

\begin{fulllineitems}
\phantomsection\label{\detokenize{index:TRAPT.DLFS.SparseGroupLasso.get_config}}
\pysigstartsignatures
\pysiglinewithargsret{\sphinxbfcode{\sphinxupquote{get\_config}}}{}{}
\pysigstopsignatures
\sphinxAtStartPar
Returns the config of the regularizer.

\sphinxAtStartPar
An regularizer config is a Python dictionary (serializable)
containing all configuration parameters of the regularizer.
The same regularizer can be reinstantiated later
(without any saved state) from this configuration.

\sphinxAtStartPar
This method is optional if you are just training and executing models,
exporting to and from SavedModels, or using weight checkpoints.

\sphinxAtStartPar
This method is required for Keras \sphinxtitleref{model\_to\_estimator}, saving and
loading models to HDF5 formats, Keras model cloning, some visualization
utilities, and exporting models to and from JSON.
\begin{quote}\begin{description}
\sphinxlineitem{Returns}
\sphinxAtStartPar
Python dictionary.

\end{description}\end{quote}

\end{fulllineitems}


\end{fulllineitems}

\index{seed\_tensorflow() (in module TRAPT.DLFS)@\spxentry{seed\_tensorflow()}\spxextra{in module TRAPT.DLFS}}

\begin{fulllineitems}
\phantomsection\label{\detokenize{index:TRAPT.DLFS.seed_tensorflow}}
\pysigstartsignatures
\pysiglinewithargsret{\sphinxcode{\sphinxupquote{TRAPT.DLFS.}}\sphinxbfcode{\sphinxupquote{seed\_tensorflow}}}{\emph{\DUrole{n}{seed}\DUrole{o}{=}\DUrole{default_value}{2023}}}{}
\pysigstopsignatures
\end{fulllineitems}



\chapter{TRAPT.DLVGAE module}
\label{\detokenize{index:module-TRAPT.DLVGAE}}\label{\detokenize{index:trapt-dlvgae-module}}\index{module@\spxentry{module}!TRAPT.DLVGAE@\spxentry{TRAPT.DLVGAE}}\index{TRAPT.DLVGAE@\spxentry{TRAPT.DLVGAE}!module@\spxentry{module}}\index{CVAE (class in TRAPT.DLVGAE)@\spxentry{CVAE}\spxextra{class in TRAPT.DLVGAE}}

\begin{fulllineitems}
\phantomsection\label{\detokenize{index:TRAPT.DLVGAE.CVAE}}
\pysigstartsignatures
\pysiglinewithargsret{\sphinxbfcode{\sphinxupquote{class\DUrole{w}{  }}}\sphinxcode{\sphinxupquote{TRAPT.DLVGAE.}}\sphinxbfcode{\sphinxupquote{CVAE}}}{\emph{\DUrole{n}{input\_dim}}, \emph{\DUrole{n}{condition\_dim}}, \emph{\DUrole{n}{h\_dim}}, \emph{\DUrole{n}{z\_dim}}}{}
\pysigstopsignatures
\sphinxAtStartPar
Bases: \sphinxcode{\sphinxupquote{Module}}
\index{forward() (TRAPT.DLVGAE.CVAE method)@\spxentry{forward()}\spxextra{TRAPT.DLVGAE.CVAE method}}

\begin{fulllineitems}
\phantomsection\label{\detokenize{index:TRAPT.DLVGAE.CVAE.forward}}
\pysigstartsignatures
\pysiglinewithargsret{\sphinxbfcode{\sphinxupquote{forward}}}{\emph{\DUrole{n}{x}}}{}
\pysigstopsignatures
\sphinxAtStartPar
Defines the computation performed at every call.

\sphinxAtStartPar
Should be overridden by all subclasses.

\begin{sphinxadmonition}{note}{Note:}
\sphinxAtStartPar
Although the recipe for forward pass needs to be defined within
this function, one should call the \sphinxcode{\sphinxupquote{Module}} instance afterwards
instead of this since the former takes care of running the
registered hooks while the latter silently ignores them.
\end{sphinxadmonition}

\end{fulllineitems}

\index{kl\_div() (TRAPT.DLVGAE.CVAE method)@\spxentry{kl\_div()}\spxextra{TRAPT.DLVGAE.CVAE method}}

\begin{fulllineitems}
\phantomsection\label{\detokenize{index:TRAPT.DLVGAE.CVAE.kl_div}}
\pysigstartsignatures
\pysiglinewithargsret{\sphinxbfcode{\sphinxupquote{kl\_div}}}{}{}
\pysigstopsignatures
\end{fulllineitems}

\index{predict\_h() (TRAPT.DLVGAE.CVAE method)@\spxentry{predict\_h()}\spxextra{TRAPT.DLVGAE.CVAE method}}

\begin{fulllineitems}
\phantomsection\label{\detokenize{index:TRAPT.DLVGAE.CVAE.predict_h}}
\pysigstartsignatures
\pysiglinewithargsret{\sphinxbfcode{\sphinxupquote{predict\_h}}}{\emph{\DUrole{n}{x}}}{}
\pysigstopsignatures
\end{fulllineitems}

\index{reparametrize() (TRAPT.DLVGAE.CVAE method)@\spxentry{reparametrize()}\spxextra{TRAPT.DLVGAE.CVAE method}}

\begin{fulllineitems}
\phantomsection\label{\detokenize{index:TRAPT.DLVGAE.CVAE.reparametrize}}
\pysigstartsignatures
\pysiglinewithargsret{\sphinxbfcode{\sphinxupquote{reparametrize}}}{\emph{\DUrole{n}{mu}}, \emph{\DUrole{n}{logstd}}}{}
\pysigstopsignatures\begin{quote}\begin{description}
\sphinxlineitem{Return type}
\sphinxAtStartPar
\sphinxcode{\sphinxupquote{Tensor}}

\end{description}\end{quote}

\end{fulllineitems}

\index{training (TRAPT.DLVGAE.CVAE attribute)@\spxentry{training}\spxextra{TRAPT.DLVGAE.CVAE attribute}}

\begin{fulllineitems}
\phantomsection\label{\detokenize{index:TRAPT.DLVGAE.CVAE.training}}
\pysigstartsignatures
\pysigline{\sphinxbfcode{\sphinxupquote{training}}\sphinxbfcode{\sphinxupquote{\DUrole{p}{:}\DUrole{w}{  }\sphinxcode{\sphinxupquote{bool}}}}}
\pysigstopsignatures
\end{fulllineitems}


\end{fulllineitems}

\index{CalcSTM (class in TRAPT.DLVGAE)@\spxentry{CalcSTM}\spxextra{class in TRAPT.DLVGAE}}

\begin{fulllineitems}
\phantomsection\label{\detokenize{index:TRAPT.DLVGAE.CalcSTM}}
\pysigstartsignatures
\pysiglinewithargsret{\sphinxbfcode{\sphinxupquote{class\DUrole{w}{  }}}\sphinxcode{\sphinxupquote{TRAPT.DLVGAE.}}\sphinxbfcode{\sphinxupquote{CalcSTM}}}{\emph{\DUrole{n}{RP\_Matrix}}, \emph{\DUrole{n}{type}}, \emph{\DUrole{n}{checkpoint\_path}}, \emph{\DUrole{n}{device}\DUrole{o}{=}\DUrole{default_value}{\textquotesingle{}cuda\textquotesingle{}}}}{}
\pysigstopsignatures
\sphinxAtStartPar
Bases: \sphinxcode{\sphinxupquote{object}}

\sphinxAtStartPar
D\sphinxhyphen{}RP model network reconstruction module.

\sphinxAtStartPar
Parameters:
RP\_Matrix : TRAPT.Tools.RP\_Matrix
\begin{quote}

\sphinxAtStartPar
TR\sphinxhyphen{}RP matrix and Epi\sphinxhyphen{}RP matrix.
\end{quote}
\begin{description}
\sphinxlineitem{type}{[}str{]}
\sphinxAtStartPar
Epi\sphinxhyphen{}RP type.

\sphinxlineitem{checkpoint\_path}{[}str{]}
\sphinxAtStartPar
Model save path.

\sphinxlineitem{device}{[}str, optional{]}
\sphinxAtStartPar
cpu/cuda.

\end{description}
\index{get\_cos\_similar\_matrix() (TRAPT.DLVGAE.CalcSTM static method)@\spxentry{get\_cos\_similar\_matrix()}\spxextra{TRAPT.DLVGAE.CalcSTM static method}}

\begin{fulllineitems}
\phantomsection\label{\detokenize{index:TRAPT.DLVGAE.CalcSTM.get_cos_similar_matrix}}
\pysigstartsignatures
\pysiglinewithargsret{\sphinxbfcode{\sphinxupquote{static\DUrole{w}{  }}}\sphinxbfcode{\sphinxupquote{get\_cos\_similar\_matrix}}}{\emph{\DUrole{n}{m1}}, \emph{\DUrole{n}{m2}}}{}
\pysigstopsignatures
\sphinxAtStartPar
Matrix cosine similarity calculation.

\end{fulllineitems}

\index{get\_edge\_index() (TRAPT.DLVGAE.CalcSTM method)@\spxentry{get\_edge\_index()}\spxextra{TRAPT.DLVGAE.CalcSTM method}}

\begin{fulllineitems}
\phantomsection\label{\detokenize{index:TRAPT.DLVGAE.CalcSTM.get_edge_index}}
\pysigstartsignatures
\pysiglinewithargsret{\sphinxbfcode{\sphinxupquote{get\_edge\_index}}}{\emph{\DUrole{n}{A}}, \emph{\DUrole{n}{B}}, \emph{\DUrole{n}{n}\DUrole{o}{=}\DUrole{default_value}{10}}}{}
\pysigstopsignatures
\sphinxAtStartPar
Construct a heterogeneous network.

\sphinxAtStartPar
Parameters:
A : anndata.AnnData
\begin{quote}

\sphinxAtStartPar
TR\sphinxhyphen{}RP matrix.
\end{quote}
\begin{description}
\sphinxlineitem{B}{[}anndata.AnnData{]}
\sphinxAtStartPar
Epi\sphinxhyphen{}RP matrix.

\sphinxlineitem{n}{[}int{]}
\sphinxAtStartPar
Number of nearest neighbors for TR.

\end{description}
\begin{quote}\begin{description}
\sphinxlineitem{Returns}
\sphinxAtStartPar
TR\sphinxhyphen{}Epi heterogeneous network.

\end{description}\end{quote}

\end{fulllineitems}

\index{init\_cvae() (TRAPT.DLVGAE.CalcSTM method)@\spxentry{init\_cvae()}\spxextra{TRAPT.DLVGAE.CalcSTM method}}

\begin{fulllineitems}
\phantomsection\label{\detokenize{index:TRAPT.DLVGAE.CalcSTM.init_cvae}}
\pysigstartsignatures
\pysiglinewithargsret{\sphinxbfcode{\sphinxupquote{init\_cvae}}}{}{}
\pysigstopsignatures
\sphinxAtStartPar
D\sphinxhyphen{}RP teacher model training function.

\end{fulllineitems}

\index{init\_vgae() (TRAPT.DLVGAE.CalcSTM method)@\spxentry{init\_vgae()}\spxextra{TRAPT.DLVGAE.CalcSTM method}}

\begin{fulllineitems}
\phantomsection\label{\detokenize{index:TRAPT.DLVGAE.CalcSTM.init_vgae}}
\pysigstartsignatures
\pysiglinewithargsret{\sphinxbfcode{\sphinxupquote{init\_vgae}}}{\emph{\DUrole{n}{h}}, \emph{\DUrole{n}{use\_kd}}}{}
\pysigstopsignatures
\sphinxAtStartPar
D\sphinxhyphen{}RP student model training function.

\sphinxAtStartPar
Parameters:
h : torch.Tensor
\begin{quote}

\sphinxAtStartPar
Potential representation of the D\sphinxhyphen{}RP teacher model.
\end{quote}
\begin{description}
\sphinxlineitem{use\_kd}{[}bool{]}
\sphinxAtStartPar
Utilize knowledge distillation.

\end{description}

\end{fulllineitems}

\index{recon\_loss() (TRAPT.DLVGAE.CalcSTM method)@\spxentry{recon\_loss()}\spxextra{TRAPT.DLVGAE.CalcSTM method}}

\begin{fulllineitems}
\phantomsection\label{\detokenize{index:TRAPT.DLVGAE.CalcSTM.recon_loss}}
\pysigstartsignatures
\pysiglinewithargsret{\sphinxbfcode{\sphinxupquote{recon\_loss}}}{\emph{\DUrole{n}{z}}, \emph{\DUrole{n}{data}}, \emph{\DUrole{n}{norm}}, \emph{\DUrole{n}{weight}}}{}
\pysigstopsignatures
\sphinxAtStartPar
Variational Gaussian Autoencoder (VGAE) reconstruction loss.

\end{fulllineitems}

\index{run() (TRAPT.DLVGAE.CalcSTM method)@\spxentry{run()}\spxextra{TRAPT.DLVGAE.CalcSTM method}}

\begin{fulllineitems}
\phantomsection\label{\detokenize{index:TRAPT.DLVGAE.CalcSTM.run}}
\pysigstartsignatures
\pysiglinewithargsret{\sphinxbfcode{\sphinxupquote{run}}}{\emph{\DUrole{n}{use\_kd}\DUrole{o}{=}\DUrole{default_value}{True}}}{}
\pysigstopsignatures
\end{fulllineitems}

\index{save\_graph() (TRAPT.DLVGAE.CalcSTM method)@\spxentry{save\_graph()}\spxextra{TRAPT.DLVGAE.CalcSTM method}}

\begin{fulllineitems}
\phantomsection\label{\detokenize{index:TRAPT.DLVGAE.CalcSTM.save_graph}}
\pysigstartsignatures
\pysiglinewithargsret{\sphinxbfcode{\sphinxupquote{save\_graph}}}{}{}
\pysigstopsignatures
\end{fulllineitems}

\index{sparse\_to\_tensor() (TRAPT.DLVGAE.CalcSTM static method)@\spxentry{sparse\_to\_tensor()}\spxextra{TRAPT.DLVGAE.CalcSTM static method}}

\begin{fulllineitems}
\phantomsection\label{\detokenize{index:TRAPT.DLVGAE.CalcSTM.sparse_to_tensor}}
\pysigstartsignatures
\pysiglinewithargsret{\sphinxbfcode{\sphinxupquote{static\DUrole{w}{  }}}\sphinxbfcode{\sphinxupquote{sparse\_to\_tensor}}}{\emph{\DUrole{n}{data}}, \emph{\DUrole{n}{type}\DUrole{o}{=}\DUrole{default_value}{\textquotesingle{}sparse\textquotesingle{}}}}{}
\pysigstopsignatures
\end{fulllineitems}


\end{fulllineitems}

\index{InnerProductDecoderWeight (class in TRAPT.DLVGAE)@\spxentry{InnerProductDecoderWeight}\spxextra{class in TRAPT.DLVGAE}}

\begin{fulllineitems}
\phantomsection\label{\detokenize{index:TRAPT.DLVGAE.InnerProductDecoderWeight}}
\pysigstartsignatures
\pysiglinewithargsret{\sphinxbfcode{\sphinxupquote{class\DUrole{w}{  }}}\sphinxcode{\sphinxupquote{TRAPT.DLVGAE.}}\sphinxbfcode{\sphinxupquote{InnerProductDecoderWeight}}}{\emph{\DUrole{n}{A\_e}}, \emph{\DUrole{o}{*}\DUrole{n}{args}}, \emph{\DUrole{o}{**}\DUrole{n}{kwargs}}}{}
\pysigstopsignatures
\sphinxAtStartPar
Bases: \sphinxcode{\sphinxupquote{InnerProductDecoder}}
\index{forward() (TRAPT.DLVGAE.InnerProductDecoderWeight method)@\spxentry{forward()}\spxextra{TRAPT.DLVGAE.InnerProductDecoderWeight method}}

\begin{fulllineitems}
\phantomsection\label{\detokenize{index:TRAPT.DLVGAE.InnerProductDecoderWeight.forward}}
\pysigstartsignatures
\pysiglinewithargsret{\sphinxbfcode{\sphinxupquote{forward}}}{\emph{\DUrole{n}{z}}, \emph{\DUrole{n}{edge\_index}\DUrole{o}{=}\DUrole{default_value}{None}}, \emph{\DUrole{n}{sigmoid}\DUrole{o}{=}\DUrole{default_value}{True}}}{}
\pysigstopsignatures
\sphinxAtStartPar
Decodes the latent variables \sphinxcode{\sphinxupquote{z}} into edge probabilities for
the given node\sphinxhyphen{}pairs \sphinxcode{\sphinxupquote{edge\_index}}.
\begin{quote}\begin{description}
\sphinxlineitem{Return type}
\sphinxAtStartPar
\sphinxcode{\sphinxupquote{Tensor}}

\sphinxlineitem{Parameters}\begin{itemize}
\item {} 
\sphinxAtStartPar
\sphinxstyleliteralstrong{\sphinxupquote{z}} (\sphinxstyleliteralemphasis{\sphinxupquote{torch.Tensor}}) \textendash{} The latent space \(\mathbf{Z}\).

\item {} 
\sphinxAtStartPar
\sphinxstyleliteralstrong{\sphinxupquote{sigmoid}} (\sphinxstyleliteralemphasis{\sphinxupquote{bool}}\sphinxstyleliteralemphasis{\sphinxupquote{, }}\sphinxstyleliteralemphasis{\sphinxupquote{optional}}) \textendash{} If set to \sphinxcode{\sphinxupquote{False}}, does not apply
the logistic sigmoid function to the output.
(default: \sphinxcode{\sphinxupquote{True}})

\end{itemize}

\end{description}\end{quote}

\end{fulllineitems}

\index{training (TRAPT.DLVGAE.InnerProductDecoderWeight attribute)@\spxentry{training}\spxextra{TRAPT.DLVGAE.InnerProductDecoderWeight attribute}}

\begin{fulllineitems}
\phantomsection\label{\detokenize{index:TRAPT.DLVGAE.InnerProductDecoderWeight.training}}
\pysigstartsignatures
\pysigline{\sphinxbfcode{\sphinxupquote{training}}\sphinxbfcode{\sphinxupquote{\DUrole{p}{:}\DUrole{w}{  }\sphinxcode{\sphinxupquote{bool}}}}}
\pysigstopsignatures
\end{fulllineitems}


\end{fulllineitems}

\index{VariationalGCNEncoder (class in TRAPT.DLVGAE)@\spxentry{VariationalGCNEncoder}\spxextra{class in TRAPT.DLVGAE}}

\begin{fulllineitems}
\phantomsection\label{\detokenize{index:TRAPT.DLVGAE.VariationalGCNEncoder}}
\pysigstartsignatures
\pysiglinewithargsret{\sphinxbfcode{\sphinxupquote{class\DUrole{w}{  }}}\sphinxcode{\sphinxupquote{TRAPT.DLVGAE.}}\sphinxbfcode{\sphinxupquote{VariationalGCNEncoder}}}{\emph{\DUrole{n}{in\_channels}}, \emph{\DUrole{n}{h\_dim}}, \emph{\DUrole{n}{z\_dim}}}{}
\pysigstopsignatures
\sphinxAtStartPar
Bases: \sphinxcode{\sphinxupquote{Module}}
\index{forward() (TRAPT.DLVGAE.VariationalGCNEncoder method)@\spxentry{forward()}\spxextra{TRAPT.DLVGAE.VariationalGCNEncoder method}}

\begin{fulllineitems}
\phantomsection\label{\detokenize{index:TRAPT.DLVGAE.VariationalGCNEncoder.forward}}
\pysigstartsignatures
\pysiglinewithargsret{\sphinxbfcode{\sphinxupquote{forward}}}{\emph{\DUrole{n}{x}}, \emph{\DUrole{n}{edge\_index}}}{}
\pysigstopsignatures
\sphinxAtStartPar
Defines the computation performed at every call.

\sphinxAtStartPar
Should be overridden by all subclasses.

\begin{sphinxadmonition}{note}{Note:}
\sphinxAtStartPar
Although the recipe for forward pass needs to be defined within
this function, one should call the \sphinxcode{\sphinxupquote{Module}} instance afterwards
instead of this since the former takes care of running the
registered hooks while the latter silently ignores them.
\end{sphinxadmonition}

\end{fulllineitems}

\index{predict\_h() (TRAPT.DLVGAE.VariationalGCNEncoder method)@\spxentry{predict\_h()}\spxextra{TRAPT.DLVGAE.VariationalGCNEncoder method}}

\begin{fulllineitems}
\phantomsection\label{\detokenize{index:TRAPT.DLVGAE.VariationalGCNEncoder.predict_h}}
\pysigstartsignatures
\pysiglinewithargsret{\sphinxbfcode{\sphinxupquote{predict\_h}}}{\emph{\DUrole{n}{x}}, \emph{\DUrole{n}{edge\_index}}}{}
\pysigstopsignatures
\end{fulllineitems}

\index{training (TRAPT.DLVGAE.VariationalGCNEncoder attribute)@\spxentry{training}\spxextra{TRAPT.DLVGAE.VariationalGCNEncoder attribute}}

\begin{fulllineitems}
\phantomsection\label{\detokenize{index:TRAPT.DLVGAE.VariationalGCNEncoder.training}}
\pysigstartsignatures
\pysigline{\sphinxbfcode{\sphinxupquote{training}}\sphinxbfcode{\sphinxupquote{\DUrole{p}{:}\DUrole{w}{  }\sphinxcode{\sphinxupquote{bool}}}}}
\pysigstopsignatures
\end{fulllineitems}


\end{fulllineitems}

\index{seed\_torch() (in module TRAPT.DLVGAE)@\spxentry{seed\_torch()}\spxextra{in module TRAPT.DLVGAE}}

\begin{fulllineitems}
\phantomsection\label{\detokenize{index:TRAPT.DLVGAE.seed_torch}}
\pysigstartsignatures
\pysiglinewithargsret{\sphinxcode{\sphinxupquote{TRAPT.DLVGAE.}}\sphinxbfcode{\sphinxupquote{seed\_torch}}}{\emph{\DUrole{n}{seed}\DUrole{o}{=}\DUrole{default_value}{2023}}}{}
\pysigstopsignatures
\end{fulllineitems}



\chapter{TRAPT.Run module}
\label{\detokenize{index:module-TRAPT.Run}}\label{\detokenize{index:trapt-run-module}}\index{module@\spxentry{module}!TRAPT.Run@\spxentry{TRAPT.Run}}\index{TRAPT.Run@\spxentry{TRAPT.Run}!module@\spxentry{module}}\index{main() (in module TRAPT.Run)@\spxentry{main()}\spxextra{in module TRAPT.Run}}

\begin{fulllineitems}
\phantomsection\label{\detokenize{index:TRAPT.Run.main}}
\pysigstartsignatures
\pysiglinewithargsret{\sphinxcode{\sphinxupquote{TRAPT.Run.}}\sphinxbfcode{\sphinxupquote{main}}}{}{}
\pysigstopsignatures
\sphinxAtStartPar
TRAPT method entry function.

\end{fulllineitems}

\index{runTRAPT() (in module TRAPT.Run)@\spxentry{runTRAPT()}\spxextra{in module TRAPT.Run}}

\begin{fulllineitems}
\phantomsection\label{\detokenize{index:TRAPT.Run.runTRAPT}}
\pysigstartsignatures
\pysiglinewithargsret{\sphinxcode{\sphinxupquote{TRAPT.Run.}}\sphinxbfcode{\sphinxupquote{runTRAPT}}}{\emph{\DUrole{n}{args}}}{}
\pysigstopsignatures
\sphinxAtStartPar
TRAPT execution entry function.
\index{args (in module TRAPT.Run)@\spxentry{args}\spxextra{in module TRAPT.Run}}

\begin{fulllineitems}
\phantomsection\label{\detokenize{index:TRAPT.Run.args}}
\pysigstartsignatures
\pysigline{\sphinxcode{\sphinxupquote{TRAPT.Run.}}\sphinxbfcode{\sphinxupquote{args}}}
\pysigstopsignatures
\sphinxAtStartPar
Global parameters for TRAPT.
\begin{quote}\begin{description}
\sphinxlineitem{Type}
\sphinxAtStartPar
{\hyperref[\detokenize{index:TRAPT.Tools.Args}]{\sphinxcrossref{TRAPT.Tools.Args}}}

\end{description}\end{quote}

\end{fulllineitems}

\index{Returns (in module TRAPT.Run)@\spxentry{Returns}\spxextra{in module TRAPT.Run}}

\begin{fulllineitems}
\phantomsection\label{\detokenize{index:TRAPT.Run.Returns}}
\pysigstartsignatures
\pysigline{\sphinxcode{\sphinxupquote{TRAPT.Run.}}\sphinxbfcode{\sphinxupquote{Returns}}}
\pysigstopsignatures
\sphinxAtStartPar
A pd.DataFrame of TR activity.

\end{fulllineitems}


\end{fulllineitems}

\index{str2bool() (in module TRAPT.Run)@\spxentry{str2bool()}\spxextra{in module TRAPT.Run}}

\begin{fulllineitems}
\phantomsection\label{\detokenize{index:TRAPT.Run.str2bool}}
\pysigstartsignatures
\pysiglinewithargsret{\sphinxcode{\sphinxupquote{TRAPT.Run.}}\sphinxbfcode{\sphinxupquote{str2bool}}}{\emph{\DUrole{n}{v}}}{}
\pysigstopsignatures
\end{fulllineitems}



\chapter{TRAPT.Tools module}
\label{\detokenize{index:module-TRAPT.Tools}}\label{\detokenize{index:trapt-tools-module}}\index{module@\spxentry{module}!TRAPT.Tools@\spxentry{TRAPT.Tools}}\index{TRAPT.Tools@\spxentry{TRAPT.Tools}!module@\spxentry{module}}\index{Args (class in TRAPT.Tools)@\spxentry{Args}\spxextra{class in TRAPT.Tools}}

\begin{fulllineitems}
\phantomsection\label{\detokenize{index:TRAPT.Tools.Args}}
\pysigstartsignatures
\pysiglinewithargsret{\sphinxbfcode{\sphinxupquote{class\DUrole{w}{  }}}\sphinxcode{\sphinxupquote{TRAPT.Tools.}}\sphinxbfcode{\sphinxupquote{Args}}}{\emph{\DUrole{n}{input}}, \emph{\DUrole{n}{output}}, \emph{\DUrole{n}{library}\DUrole{o}{=}\DUrole{default_value}{\textquotesingle{}library\textquotesingle{}}}, \emph{\DUrole{n}{threads}\DUrole{o}{=}\DUrole{default_value}{16}}, \emph{\DUrole{n}{trunk\_size}\DUrole{o}{=}\DUrole{default_value}{32768}}, \emph{\DUrole{n}{background\_genes}\DUrole{o}{=}\DUrole{default_value}{6000}}, \emph{\DUrole{n}{use\_kd}\DUrole{o}{=}\DUrole{default_value}{True}}, \emph{\DUrole{n}{tr\_type}\DUrole{o}{=}\DUrole{default_value}{\textquotesingle{}all\textquotesingle{}}}, \emph{\DUrole{n}{source}\DUrole{o}{=}\DUrole{default_value}{\textquotesingle{}all\textquotesingle{}}}}{}
\pysigstopsignatures
\sphinxAtStartPar
Bases: \sphinxcode{\sphinxupquote{object}}

\sphinxAtStartPar
TRAPT Global Parameters.
\index{input (TRAPT.Tools.Args attribute)@\spxentry{input}\spxextra{TRAPT.Tools.Args attribute}}

\begin{fulllineitems}
\phantomsection\label{\detokenize{index:TRAPT.Tools.Args.input}}
\pysigstartsignatures
\pysigline{\sphinxbfcode{\sphinxupquote{input}}}
\pysigstopsignatures
\sphinxAtStartPar
Input path for the gene set.
\begin{quote}\begin{description}
\sphinxlineitem{Type}
\sphinxAtStartPar
str

\end{description}\end{quote}

\end{fulllineitems}

\index{output (TRAPT.Tools.Args attribute)@\spxentry{output}\spxextra{TRAPT.Tools.Args attribute}}

\begin{fulllineitems}
\phantomsection\label{\detokenize{index:TRAPT.Tools.Args.output}}
\pysigstartsignatures
\pysigline{\sphinxbfcode{\sphinxupquote{output}}}
\pysigstopsignatures
\sphinxAtStartPar
Output path for TRAPT results.
\begin{quote}\begin{description}
\sphinxlineitem{Type}
\sphinxAtStartPar
str

\end{description}\end{quote}

\end{fulllineitems}

\index{library (TRAPT.Tools.Args attribute)@\spxentry{library}\spxextra{TRAPT.Tools.Args attribute}}

\begin{fulllineitems}
\phantomsection\label{\detokenize{index:TRAPT.Tools.Args.library}}
\pysigstartsignatures
\pysigline{\sphinxbfcode{\sphinxupquote{library}}}
\pysigstopsignatures
\sphinxAtStartPar
Path to the background library, default is the ‘library’ path in the current directory.
\begin{quote}\begin{description}
\sphinxlineitem{Type}
\sphinxAtStartPar
str, optional

\end{description}\end{quote}

\end{fulllineitems}

\index{threads (TRAPT.Tools.Args attribute)@\spxentry{threads}\spxextra{TRAPT.Tools.Args attribute}}

\begin{fulllineitems}
\phantomsection\label{\detokenize{index:TRAPT.Tools.Args.threads}}
\pysigstartsignatures
\pysigline{\sphinxbfcode{\sphinxupquote{threads}}}
\pysigstopsignatures
\sphinxAtStartPar
Number of processes used for TRAPT inference.
\begin{quote}\begin{description}
\sphinxlineitem{Type}
\sphinxAtStartPar
int, optional

\end{description}\end{quote}

\end{fulllineitems}

\index{trunk\_size (TRAPT.Tools.Args attribute)@\spxentry{trunk\_size}\spxextra{TRAPT.Tools.Args attribute}}

\begin{fulllineitems}
\phantomsection\label{\detokenize{index:TRAPT.Tools.Args.trunk_size}}
\pysigstartsignatures
\pysigline{\sphinxbfcode{\sphinxupquote{trunk\_size}}}
\pysigstopsignatures
\sphinxAtStartPar
Size of the chunks.
\begin{quote}\begin{description}
\sphinxlineitem{Type}
\sphinxAtStartPar
int, optional

\end{description}\end{quote}

\end{fulllineitems}

\index{background\_genes (TRAPT.Tools.Args attribute)@\spxentry{background\_genes}\spxextra{TRAPT.Tools.Args attribute}}

\begin{fulllineitems}
\phantomsection\label{\detokenize{index:TRAPT.Tools.Args.background_genes}}
\pysigstartsignatures
\pysigline{\sphinxbfcode{\sphinxupquote{background\_genes}}}
\pysigstopsignatures
\sphinxAtStartPar
Number of background genes selected.
\begin{quote}\begin{description}
\sphinxlineitem{Type}
\sphinxAtStartPar
str, optional

\end{description}\end{quote}

\end{fulllineitems}

\index{use\_kd (TRAPT.Tools.Args attribute)@\spxentry{use\_kd}\spxextra{TRAPT.Tools.Args attribute}}

\begin{fulllineitems}
\phantomsection\label{\detokenize{index:TRAPT.Tools.Args.use_kd}}
\pysigstartsignatures
\pysigline{\sphinxbfcode{\sphinxupquote{use\_kd}}}
\pysigstopsignatures
\sphinxAtStartPar
Use knowledge distillation.
\begin{quote}\begin{description}
\sphinxlineitem{Type}
\sphinxAtStartPar
str, optional

\end{description}\end{quote}

\end{fulllineitems}

\index{tr\_type (TRAPT.Tools.Args attribute)@\spxentry{tr\_type}\spxextra{TRAPT.Tools.Args attribute}}

\begin{fulllineitems}
\phantomsection\label{\detokenize{index:TRAPT.Tools.Args.tr_type}}
\pysigstartsignatures
\pysigline{\sphinxbfcode{\sphinxupquote{tr\_type}}}
\pysigstopsignatures
\sphinxAtStartPar
all/tf/tcof/cr.
\begin{quote}\begin{description}
\sphinxlineitem{Type}
\sphinxAtStartPar
str, optional

\end{description}\end{quote}

\end{fulllineitems}

\index{source (TRAPT.Tools.Args attribute)@\spxentry{source}\spxextra{TRAPT.Tools.Args attribute}}

\begin{fulllineitems}
\phantomsection\label{\detokenize{index:TRAPT.Tools.Args.source}}
\pysigstartsignatures
\pysigline{\sphinxbfcode{\sphinxupquote{source}}}
\pysigstopsignatures
\sphinxAtStartPar
all/cistrome/chip\_altas/gtrd/remap/chip\sphinxhyphen{}atlas/remap/encode/geo.
\begin{quote}\begin{description}
\sphinxlineitem{Type}
\sphinxAtStartPar
str, optional

\end{description}\end{quote}

\end{fulllineitems}


\end{fulllineitems}

\index{RPMatrix (class in TRAPT.Tools)@\spxentry{RPMatrix}\spxextra{class in TRAPT.Tools}}

\begin{fulllineitems}
\phantomsection\label{\detokenize{index:TRAPT.Tools.RPMatrix}}
\pysigstartsignatures
\pysiglinewithargsret{\sphinxbfcode{\sphinxupquote{class\DUrole{w}{  }}}\sphinxcode{\sphinxupquote{TRAPT.Tools.}}\sphinxbfcode{\sphinxupquote{RPMatrix}}}{\emph{\DUrole{n}{library}}, \emph{\DUrole{n}{name}}, \emph{\DUrole{n}{to\_array}\DUrole{o}{=}\DUrole{default_value}{True}}}{}
\pysigstopsignatures
\sphinxAtStartPar
Bases: \sphinxcode{\sphinxupquote{object}}
\index{add() (TRAPT.Tools.RPMatrix method)@\spxentry{add()}\spxextra{TRAPT.Tools.RPMatrix method}}

\begin{fulllineitems}
\phantomsection\label{\detokenize{index:TRAPT.Tools.RPMatrix.add}}
\pysigstartsignatures
\pysiglinewithargsret{\sphinxbfcode{\sphinxupquote{add}}}{\emph{\DUrole{n}{data}}}{}
\pysigstopsignatures
\end{fulllineitems}

\index{binarization() (TRAPT.Tools.RPMatrix method)@\spxentry{binarization()}\spxextra{TRAPT.Tools.RPMatrix method}}

\begin{fulllineitems}
\phantomsection\label{\detokenize{index:TRAPT.Tools.RPMatrix.binarization}}
\pysigstartsignatures
\pysiglinewithargsret{\sphinxbfcode{\sphinxupquote{binarization}}}{}{}
\pysigstopsignatures
\end{fulllineitems}

\index{get\_data() (TRAPT.Tools.RPMatrix method)@\spxentry{get\_data()}\spxextra{TRAPT.Tools.RPMatrix method}}

\begin{fulllineitems}
\phantomsection\label{\detokenize{index:TRAPT.Tools.RPMatrix.get_data}}
\pysigstartsignatures
\pysiglinewithargsret{\sphinxbfcode{\sphinxupquote{get\_data}}}{}{}
\pysigstopsignatures\begin{quote}\begin{description}
\sphinxlineitem{Return type}
\sphinxAtStartPar
\sphinxcode{\sphinxupquote{AnnData}}

\end{description}\end{quote}

\end{fulllineitems}

\index{minmax\_scale() (TRAPT.Tools.RPMatrix method)@\spxentry{minmax\_scale()}\spxextra{TRAPT.Tools.RPMatrix method}}

\begin{fulllineitems}
\phantomsection\label{\detokenize{index:TRAPT.Tools.RPMatrix.minmax_scale}}
\pysigstartsignatures
\pysiglinewithargsret{\sphinxbfcode{\sphinxupquote{minmax\_scale}}}{\emph{\DUrole{n}{axis}\DUrole{o}{=}\DUrole{default_value}{1}}}{}
\pysigstopsignatures
\end{fulllineitems}

\index{norm() (TRAPT.Tools.RPMatrix method)@\spxentry{norm()}\spxextra{TRAPT.Tools.RPMatrix method}}

\begin{fulllineitems}
\phantomsection\label{\detokenize{index:TRAPT.Tools.RPMatrix.norm}}
\pysigstartsignatures
\pysiglinewithargsret{\sphinxbfcode{\sphinxupquote{norm}}}{\emph{\DUrole{n}{type}\DUrole{o}{=}\DUrole{default_value}{\textquotesingle{}l2\textquotesingle{}}}, \emph{\DUrole{n}{axis}\DUrole{o}{=}\DUrole{default_value}{1}}}{}
\pysigstopsignatures
\end{fulllineitems}

\index{standard\_scale() (TRAPT.Tools.RPMatrix method)@\spxentry{standard\_scale()}\spxextra{TRAPT.Tools.RPMatrix method}}

\begin{fulllineitems}
\phantomsection\label{\detokenize{index:TRAPT.Tools.RPMatrix.standard_scale}}
\pysigstartsignatures
\pysiglinewithargsret{\sphinxbfcode{\sphinxupquote{standard\_scale}}}{\emph{\DUrole{n}{axis}\DUrole{o}{=}\DUrole{default_value}{1}}}{}
\pysigstopsignatures
\end{fulllineitems}


\end{fulllineitems}

\index{RP\_Matrix (class in TRAPT.Tools)@\spxentry{RP\_Matrix}\spxextra{class in TRAPT.Tools}}

\begin{fulllineitems}
\phantomsection\label{\detokenize{index:TRAPT.Tools.RP_Matrix}}
\pysigstartsignatures
\pysiglinewithargsret{\sphinxbfcode{\sphinxupquote{class\DUrole{w}{  }}}\sphinxcode{\sphinxupquote{TRAPT.Tools.}}\sphinxbfcode{\sphinxupquote{RP\_Matrix}}}{\emph{\DUrole{n}{library}}}{}
\pysigstopsignatures
\sphinxAtStartPar
Bases: \sphinxcode{\sphinxupquote{object}}

\end{fulllineitems}

\index{Type (class in TRAPT.Tools)@\spxentry{Type}\spxextra{class in TRAPT.Tools}}

\begin{fulllineitems}
\phantomsection\label{\detokenize{index:TRAPT.Tools.Type}}
\pysigstartsignatures
\pysigline{\sphinxbfcode{\sphinxupquote{class\DUrole{w}{  }}}\sphinxcode{\sphinxupquote{TRAPT.Tools.}}\sphinxbfcode{\sphinxupquote{Type}}}
\pysigstopsignatures
\sphinxAtStartPar
Bases: \sphinxcode{\sphinxupquote{object}}
\index{ATAC (TRAPT.Tools.Type attribute)@\spxentry{ATAC}\spxextra{TRAPT.Tools.Type attribute}}

\begin{fulllineitems}
\phantomsection\label{\detokenize{index:TRAPT.Tools.Type.ATAC}}
\pysigstartsignatures
\pysigline{\sphinxbfcode{\sphinxupquote{ATAC}}\sphinxbfcode{\sphinxupquote{\DUrole{w}{  }\DUrole{p}{=}\DUrole{w}{  }\textquotesingle{}ATAC\textquotesingle{}}}}
\pysigstopsignatures
\end{fulllineitems}

\index{H3K27ac (TRAPT.Tools.Type attribute)@\spxentry{H3K27ac}\spxextra{TRAPT.Tools.Type attribute}}

\begin{fulllineitems}
\phantomsection\label{\detokenize{index:TRAPT.Tools.Type.H3K27ac}}
\pysigstartsignatures
\pysigline{\sphinxbfcode{\sphinxupquote{H3K27ac}}\sphinxbfcode{\sphinxupquote{\DUrole{w}{  }\DUrole{p}{=}\DUrole{w}{  }\textquotesingle{}H3K27ac\textquotesingle{}}}}
\pysigstopsignatures
\end{fulllineitems}


\end{fulllineitems}



\chapter{Module contents}
\label{\detokenize{index:module-TRAPT}}\label{\detokenize{index:module-contents}}\index{module@\spxentry{module}!TRAPT@\spxentry{TRAPT}}\index{TRAPT@\spxentry{TRAPT}!module@\spxentry{module}}

\renewcommand{\indexname}{Python Module Index}
\begin{sphinxtheindex}
\let\bigletter\sphinxstyleindexlettergroup
\bigletter{t}
\item\relax\sphinxstyleindexentry{TRAPT}\sphinxstyleindexpageref{index:\detokenize{module-TRAPT}}
\item\relax\sphinxstyleindexentry{TRAPT.CalcSampleRPMatrix}\sphinxstyleindexpageref{index:\detokenize{module-TRAPT.CalcSampleRPMatrix}}
\item\relax\sphinxstyleindexentry{TRAPT.CalcTRAUC}\sphinxstyleindexpageref{index:\detokenize{module-TRAPT.CalcTRAUC}}
\item\relax\sphinxstyleindexentry{TRAPT.CalcTRRPMatrix}\sphinxstyleindexpageref{index:\detokenize{module-TRAPT.CalcTRRPMatrix}}
\item\relax\sphinxstyleindexentry{TRAPT.CalcTRSampleRPMatrix}\sphinxstyleindexpageref{index:\detokenize{module-TRAPT.CalcTRSampleRPMatrix}}
\item\relax\sphinxstyleindexentry{TRAPT.DLFS}\sphinxstyleindexpageref{index:\detokenize{module-TRAPT.DLFS}}
\item\relax\sphinxstyleindexentry{TRAPT.DLVGAE}\sphinxstyleindexpageref{index:\detokenize{module-TRAPT.DLVGAE}}
\item\relax\sphinxstyleindexentry{TRAPT.Run}\sphinxstyleindexpageref{index:\detokenize{module-TRAPT.Run}}
\item\relax\sphinxstyleindexentry{TRAPT.Tools}\sphinxstyleindexpageref{index:\detokenize{module-TRAPT.Tools}}
\end{sphinxtheindex}

\renewcommand{\indexname}{Index}
\printindex
\end{document}